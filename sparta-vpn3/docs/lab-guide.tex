\input{header}
\begin{document}

\begin{center}
{\LARGE IPTables: Site-to-Site VPN - Lab Guide}
\vspace{0.1in}\\
\end{center}

\copyrightnotice

\section{Overview}
A VPN is virtual in that it carries information within a private network, but that information is actually transported over a public network.

A VPN is private in that the traffic is encrypted to keep the data confidential while it is transported across the public network.

OpenVPN is an open source tool that can be used to create VPN conections. It uses a custom protocol based on SSL and TLS.

Site-to-site VPNs are used to connect networks across another untrusted network such as the internet.

In a site-to-site VPN, end hosts send and receive normal unencrypted TCP/IP traffic through a VPN terminating device.

In most cases, the devices in both networks don't need to do any more steps to exchange data with the devices in the other network.

We will enhance the lab Host-to-Site VPN to establish the VPN connection between Site1-Router and the HQ-Router through the ISP network.

\section{Lab Environment}
This lab runs in the Labtainer framework,
available at http://my.nps.edu/web/c3o/labtainers.
That site includes links to a pre-built virtual machine
that has Labtainers installed, however Labtainers can
be run on any Linux host that supports Docker containers.

From your labtainer-student (~/labtainer/labtainer-student) directory start the lab using:
\begin{verbatim}
    labtainer sparta-vpn3
\end{verbatim}
\noindent A link to this lab guide will be displayed.

\section{Network Configuration}
IP addresses and routing are configured on all devices.

\begin{figure}[H]
\begin{center}
\includegraphics [width=0.8\textwidth]{labtainers-vpn3-lab-01.png}
\end{center}
\caption{Network topology for routing-basics lab}
\label{fig:topology}
\end{figure}

The network is composed of (find the main components of this network based on its topology):
\begin{itemize}
	\item
	\item
	\item
	\item
	\item
	\item
	\item
\end{itemize}

The OpenVPN application is pre-installed on the host and the server, and the OpenVPN configuration files already exist.

\section{Credentials}
\begin{itemize}
	\item \textbf{Host-11}:
	\begin{itemize}
		\item \textbf{Username:} user-11
		\item \textbf{Password:} user-11
	\end{itemize}
	\item \textbf{Host-12}:
	\begin{itemize}
		\item \textbf{Username:} user-12
		\item \textbf{Password:} user-12
	\end{itemize}
	\item \textbf{Web-Server1}:
	\begin{itemize}
		\item \textbf{Username:} web-admin
		\item \textbf{Password:} web-admin
	\end{itemize}
	\item \textbf{Web-Server2}:
	\begin{itemize}
		\item \textbf{Username:} web-admin
		\item \textbf{Password:} web-admin
	\end{itemize}
	\item \textbf{Site1-Router}:
	\begin{itemize}
		\item \textbf{Username:} admin
		\item \textbf{Password:} admin
	\end{itemize}
	\item \textbf{ISP-Router}:
	\begin{itemize}
		\item \textbf{Username:} admin
		\item \textbf{Password:} admin
	\end{itemize}
	\item \textbf{HQ-Router}:
	\begin{itemize}
		\item \textbf{Username:} admin
		\item \textbf{Password:} admin
	\end{itemize}
\end{itemize}

\section{Lab Tasks}

\subsection{Checking/Testing the Initial Configuration}\label{Checking/Testing the Initial Configuration}
Its your task to test the configuration and to report what does/not work


\subsection{Capturing the Traffic}
Lets use \textbf{tcpdump} which is a command line tool that can capture TCP/IP and other packets being transmitted or received over a network interface.
\newline
\newline
In the ISP-Router terminal, run the following command:
\begin{verbatim}
    sudo tcpdump -n -XX -i eth1
\end{verbatim}

Lets run tcpdump on Web-Server1 so we can monitor the traffic received on the server:
\begin{verbatim}
    sudo tcpdump -n -XX -i eth0
\end{verbatim}

Lets run tcpdump on Web-Server2 so we can monitor the traffic received on the server:
\begin{verbatim}
    sudo tcpdump -n -XX -i eth0
\end{verbatim}

\subsection{Configuring the VPN - HQ-Router}
\begin{itemize}
	\item Check VPN configuration

		\begin{verbatim}
				ls
		\end{verbatim}

		Notice that there are two files:

		\begin{itemize}
			\item gateway.conf: it contains the OpenVPN  configuration that will be used during establishing the VPN connection with the VPN-Gateway.
				\begin{verbatim}
						cat gateway.conf
				\end{verbatim}

				The file content is as follows:
					\begin{itemize}
						\item The first line `dev tun` is the defining the type of the device/interface that will be created during the VPN establishment. In this case, we are using tunnel interface.
						\item The second line `ifconfig 192.168.1.1 192.168.1.2` defines the IP addresses of both end of tunnel.
						\item The third line `secret static.key` points to the file where the shared static key is stored.
						\item The fourth line `push "route 10.1.2.0 255.255.255.0"` advertises to the vpn-client that this network can be reached through the VPN-tunnel.
						\item The fifth line `route 10.1.1.0 255.255.255.0 192.168.1.1 1` adds a new routing record to route thr traffic to this network through the tunnel interface. The `1` at the end of the line gives this route a higher priority than all other previous routes to the same network.
					\end{itemize}
			\item static.key: it contains the same secret key that exists on the client and will be used to establish the VPN connection with the hq-router.
				\begin{verbatim}
						cat static.conf
				\end{verbatim}

		\end{itemize}

	\item Establishing the VPN connection.
		\begin{verbatim}
				sudo openvpn --config gateway.conf --daemon
		\end{verbatim}
\end{itemize}

\subsection{Configuring the VPN - Site1-Router Side}
\begin{itemize}
	\item Check VPN configuration

		\begin{verbatim}
				ls
		\end{verbatim}

		Notice that there are two files:

		\begin{itemize}
			\item site1.conf: it contains the OpenVPN  configuration that will be used during establishing the VPN connection with the VPN-Gateway.
				\begin{verbatim}
						cat site1.conf
				\end{verbatim}

				The file content is as follows:
					\begin{itemize}
						\item The first line `remote 200.200.200.2` is the OpenVPN Gateway that it will connect to.
						\item The second line `dev tun` is the defining the type of the device/interface that will be created during the VPN establishment. In this case, we are using tunnel interface.
						\item The third line `ifconfig 192.168.1.1 192.168.1.2` defines the IP addresses of both end of tunnel.
						\item The fourth line `secret static.key` points to the file where the shared static key is stored.
						\item The fifth line `push "route 10.1.1.0 255.255.255.0"` advertises to the vpn-gateway that this network can be reached through the VPN-tunnel.
						\item The sixth line `route 10.1.2.0 255.255.255.0 192.168.1.1` adds a new routing record to route thr traffic to this network through the tunnel interface. The `1` at the end of the line gives this route a higher priority than all other previous routes to the same network.
					\end{itemize}
			\item static.key: it contains the secret key that will be used to establish the VPN connection with the VPN-Gateway. This is a shared key, that means the VPN-Gateway should have the exact key in order to successfully establish the connection.
				\begin{verbatim}
						cat static.conf
				\end{verbatim}

		\end{itemize}

	\item Start the VPN connection on the VPN-Gateway.
		\begin{verbatim}
				sudo openvpn --config site1.conf --daemon
		\end{verbatim}
\end{itemize}


\section{Testing the connectivity}
\begin{itemize}
	\item On HQ-Router (Check IP configuration and network interfaces)
	\begin{verbatim}
	    ip addr
	\end{verbatim}

	What is the result ? How many interfaces on the this machine ?

	\item On Site1-Router (Check IP configuration and network interfaces)
	\begin{verbatim}
	    ip addr
	\end{verbatim}

	What is the result ? How many interfaces on the this machine ?

	\item On Host-11, Host-12 (Ping Host-11,Host-12 -$>$ Web-Server1, Web-Server2)
	\begin{verbatim}
	    ping 10.1.2.1
			wget 10.1.2.1
			ping 10.1.2.2
			wget 10.1.2.2
	\end{verbatim}

	What is the result ? Why do we used the private IP addresses not the tunnel ones ?

	\item Check the router terminal to see whether you still can find unencrypted text in the packets captured by the router.

	Check whether you can see the IP addresses (10.1.1.1, 10.1.2.1 or 10.1.2.2). Why can't you see any of them ?

	\item Check the server terminal and check which IP addresses you can see in the log. Can you see the IP address 10.1.1.1 ? Why ?

\end{itemize}

\section{Submission}
After finishing the lab, go to the terminal on your Linux system that was used to start the lab and type:
\begin{verbatim}
    stoplab sparta-vpn3
\end{verbatim}
When you stop the lab, the system will display a path to the zipped lab results on your Linux system.  Provide that file to
your instructor, e.g., via the Sakai site.

\end{document}
